\documentclass[12pt]{report}
\usepackage{geometry}
\usepackage{titling}

\geometry{margin=1in}

\newcommand{\subtitle}[1]{%
	\posttitle{%
		\par\end{center}
		\begin{center}\large#1\end{center}
		\vskip0.5em}%
}

\title{Application Domain Study}
\subtitle{CS 383 - Homework 1 - Group 3}
\author{Mason Fabel, Ronald Rodriguez, NAMES} % TODO: Everyone add your name to this list
\date{\today}

\begin{document}

\maketitle

\chapter{Related Applications}

% TODO: Each person needs a section here.

\begin{section}{Nethack - Mason Fabel}
For the application domain study, I chose to study the classic roguelike
game Nethack. I chose this title for a number of reasons. First of all,
the roguelike genre is, if not the direct predecesor, the inspiration for
the majority of tile-based dungeon crawling games, among others. This being
so, I chose Nethack as it a mature and heavily played roguelike, being one
of a handful of big-name roguelike games.

For my study, I am going to play a session of Nethack and record a
condensed version of my adventures. I will then discuss the elements that
I think can be used successfully in our project, followed by the elements
that I think can not.

\begin{subsection}{Gameplay}
Nethack opens with a pretty straightforward selection of menus asking the
player for their class, race, and gender. I selected a human male Tourist,
as I am merely making a small tour of this game for this study. A short,
three paragraph intro is dropped on me, telling me to descend the
dungeon and fetch some amulet.

And that's where the handholding stops. I'm dropped into the game with the
oh-so-helpful message "Be careful!", and nothing else. Some quick fiddling
will reveal that the various keypresses correspond to different actions,
and that pressing '?' will bring up some very cryptic help. Fortunately,
this is far from my first time in Nethack, so I'm over the gigantic
learning curve needed to even make sense of the screen and perform basic
actions.

Thus, off I go, looking for the amulet. The game itself was pretty simple
to watch. Explore the map, look for the stairs to go deeper, run away from
monsters until my pet kills them and I loot the corpses. In this manner I
cleared a few levels and then called it a day.
\end{subsection}

\begin{subsection}{The Good}
There are a lot of good, fun elements to Nethack, even if I didn't come
across too many of them in the short period I had to play and study it.
Some of these features are a good match for the project we are making,
while others are not so good.

The first element of note which could be put to good use in our project
is the fact that each dungeon level is randomly generated, and so are
the items, to a certain extent. What Nethack does is layout a general
level generation scheme (rectangular rooms connected by corridors during
the early levels), and then randomly create floors which follow those
rules. Thus, the game isn't at the complete mercy of the random generator,
but can set and follow certain expectations while still providing variety
and a unique experience each game.

Furthermore, the item's are partially randomized. While each game of
Nethack will have the same potion of acid, ring of levitation, or other
items, but each game the effect of, say, the red potion is different.
Play once, the red potion is healing. Play again, it's hallucination, or
perhaps grease. While this particular mechanic hinges on the fact that a
major part of a game of Nethack is discovering what your items are and
what they do, the general principle of using controlled randomness to add
variety to different games is praiseworthy, and should in my opinion be
used throughout our final project.
\end{subsection}

\begin{subsection}{The Bad}
While Nethack is a fantastic game, some of it's features would translate
extremely poorly over to our project. After all, Nethack is a single-player
dungeon crawler with an emphasis on mechanical depth, and our project
is a cooperative multiplayer game, which calls for what can be very
different features.

The most obvious feature to discard would the be the interface. While
Nethack functions quite well once you grow accustomed to it, the controls
are very obtuse and extremely difficult to learn. To further complicate
the issue, most of the information regarding what is happening and all of
the information regarding what actions the player can take are completely
hidden, unless one already knows a complicated maze of menus and actions.
When we build our project, we should make it very clear what is currently
happening and what actions the player can respond with.

Another feature of Nethack that should definitely be dropped from our
project is permanent player death. In Nethack, if you die then your game is
over, permanently, and you must start again. While that contributes in a
large way to the charm of the game, this is completely out of place in a
multiplayer game, where a single mistake could force the player out of an
otherwise pleasant game. Furthermore, when playing with other people it
is very often fun to compare progress and try to one-up your friends.
Having the game threaten to wipe that out forever is typically not a good
feature for such games.
\end{subsection}

\end{section}

\begin{section}{Teleglitch - Ronald Rodriguez}
For my application domain study, I chose to study Teleglitch. Teleglitch 
is a rogue-like top-down shooter/slasher action game. I chose this game 
as I felt it was somewhat similar to our goal of a tile based rpg without 
"waiting" battles.

I will complete the first few levels of Teleglitch and record my thoughts 
about which features I feel would and would not work for our 
game. 

\begin{subsection}{gameplay}
From the start menu of Teleglitch, the player can choose to play, read a 
tutorial, or choose a level they have previously completed. Once the 
user selects play/continue, they are dropped into the world. Players can move 
the character around the screen using standard keyboard/mouse controls. 

The way levels are designed and generated in Teleglitch is interesting. Not 
only are levels randomly generated each time the user plays, but there is a 
neat mechanic where most of the level remains hidden until your character 
actually sees that section by navigating over to it. This causes great 
moments of surprise when you travel to a new part of the level only to be 
met by a horde of enemies. 

The core of the gameplay comes in the form of killing enemies using various 
weapons. The character begins the game with a sword that he can use to melee 
enemies with. Other weapons like guns and bombs are obtained via loot drops 
from downed enemies and item boxes. One of the main mechanics of the game comes 
in the form of being able to combine certain loot items into more useful weapons. 
For instance, you may find some nails in an item box at one point, and then a bit 
later find a baseball bat after you kill an enemy. The game allows you to simply 
combine these two items into a nail-bat which gives it a damage boost over a 
standard baseball bat. There are multitudes of these potential item combinations. 

The game also incorporates elements of the survival horror genre in the form of 
ammo being somewhat of a rarity, encouraging the player to think very carefully 
about when they choose to use their guns insted of their melee weapons.
\end{subsection}

\begin{subsection}{The Good}
The shooting mechanics in Teleglitch are exceptional. When a gun is fired at an 
enemy, a noticable "blur" effect happens to the character and the bullet itself. 
This effect adds a palpable "weight" to the shots, making them visceral, impactful, 
and highly satisfying. 

The loot combining element is an excellent mechanic. It causes players to think 
critically about the potential ways they can combine the multitude of items in their 
inventory. A great sense of satisfaction comes when you think of a way to combine 
two things and the game allows you to do so exactly as you envisioned. 

The level designs are very cool, having a very industrial, apocalyptic aesthetic. 
The levels are somewhat minimal in design, which lends well to the game overall, as 
there is rarely any unnecessary clutter obstructing you from attacking your foes or 
getting around the environment. 

While it has been a complaint of many industry reviewers, I think the low-fi look of 
the game overall is quite charming. A game like this doesn't need to have super high-res 
graphics. I felt it would add almost nothing to the game at all. 

\end{subsection}

\begin{subsection}{The Bad}
The application of "ammo rarity" in this game only ended up frustrating me, rather than 
giving me a feeling of tension. With Teleglitch, where the shooting is easily the highlight of 
the entire game, implementing a mechanic that lessens the amount of times you will use that 
awesome gunplay seems silly to me. 

In Teleglitch, when you want to see the whole map of the level you are currently on, you need 
to press a button that actually zooms out from the current room your in and shows you the whole 
map. I felt this feature broke the immersion of the game. I think an onscreen mini-map that 
could be toggled on and off would have been a much better choice both for Teleglitch and for 
our game.

\end{subsection}

\end{section}











\end{document}
