\documentclass[12pt]{report}
\usepackage{geometry}
\usepackage{titling}

\geometry{margin=1in}

\newcommand{\subtitle}[1]{%
	\posttitle{%
		\par\end{center}
		\begin{center}\large#1\end{center}
		\vskip0.5em}%
}

\title{Application Domain Study}
\subtitle{CS 383 - Homework 1 - Group 3}
\author{Mason Fabel, Ronald Rodriguez, Lance Wells, Morgan Holbart, Zachary Yama, NAMES} % TODO: Everyone add your name to this list
\date{\today}

\begin{document}

\maketitle

\chapter{Related Applications}

% TODO: Each person needs a section here.

\begin{section}{Nethack - Mason Fabel}
For the application domain study, I chose to study the classic roguelike
game Nethack. I chose this title for a number of reasons. First of all,
the roguelike genre is, if not the direct predecesor, the inspiration for
the majority of tile-based dungeon crawling games, among others. This being
so, I chose Nethack as it a mature and heavily played roguelike, being one
of a handful of big-name roguelike games.

For my study, I am going to play a session of Nethack and record a
condensed version of my adventures. I will then discuss the elements that
I think can be used successfully in our project, followed by the elements
that I think can not.

\begin{subsection}{Gameplay}
Nethack opens with a pretty straightforward selection of menus asking the
player for their class, race, and gender. I selected a human male Tourist,
as I am merely making a small tour of this game for this study. A short,
three paragraph intro is dropped on me, telling me to descend the
dungeon and fetch some amulet.

And that's where the handholding stops. I'm dropped into the game with the
oh-so-helpful message "Be careful!", and nothing else. Some quick fiddling
will reveal that the various keypresses correspond to different actions,
and that pressing '?' will bring up some very cryptic help. Fortunately,
this is far from my first time in Nethack, so I'm over the gigantic
learning curve needed to even make sense of the screen and perform basic
actions.

Thus, off I go, looking for the amulet. The game itself was pretty simple
to watch. Explore the map, look for the stairs to go deeper, run away from
monsters until my pet kills them and I loot the corpses. In this manner I
cleared a few levels and then called it a day.
\end{subsection}

\begin{subsection}{The Good}
There are a lot of good, fun elements to Nethack, even if I didn't come
across too many of them in the short period I had to play and study it.
Some of these features are a good match for the project we are making,
while others are not so good.

The first element of note which could be put to good use in our project
is the fact that each dungeon level is randomly generated, and so are
the items, to a certain extent. What Nethack does is layout a general
level generation scheme (rectangular rooms connected by corridors during
the early levels), and then randomly create floors which follow those
rules. Thus, the game isn't at the complete mercy of the random generator,
but can set and follow certain expectations while still providing variety
and a unique experience each game.

Furthermore, the item's are partially randomized. While each game of
Nethack will have the same potion of acid, ring of levitation, or other
items, but each game the effect of, say, the red potion is different.
Play once, the red potion is healing. Play again, it's hallucination, or
perhaps grease. While this particular mechanic hinges on the fact that a
major part of a game of Nethack is discovering what your items are and
what they do, the general principle of using controlled randomness to add
variety to different games is praiseworthy, and should in my opinion be
used throughout our final project.
\end{subsection}

\begin{subsection}{The Bad}
While Nethack is a fantastic game, some of it's features would translate
extremely poorly over to our project. After all, Nethack is a single-player
dungeon crawler with an emphasis on mechanical depth, and our project
is a cooperative multiplayer game, which calls for what can be very
different features.

The most obvious feature to discard would the be the interface. While
Nethack functions quite well once you grow accustomed to it, the controls
are very obtuse and extremely difficult to learn. To further complicate
the issue, most of the information regarding what is happening and all of
the information regarding what actions the player can take are completely
hidden, unless one already knows a complicated maze of menus and actions.
When we build our project, we should make it very clear what is currently
happening and what actions the player can respond with.

Another feature of Nethack that should definitely be dropped from our
project is permanent player death. In Nethack, if you die then your game is
over, permanently, and you must start again. While that contributes in a
large way to the charm of the game, this is completely out of place in a
multiplayer game, where a single mistake could force the player out of an
otherwise pleasant game. Furthermore, when playing with other people it
is very often fun to compare progress and try to one-up your friends.
Having the game threaten to wipe that out forever is typically not a good
feature for such games.
\end{subsection}

\end{section}

\begin{section}{Teleglitch - Ronald Rodriguez}
For my application domain study, I chose to study Teleglitch. Teleglitch 
is a rogue-like top-down shooter/slasher action game. I chose this game 
as I felt it was somewhat similar to our goal of a tile based rpg without 
"waiting" battles.

I will complete the first few levels of Teleglitch and record my thoughts 
about which features I feel would and would not work for our 
game. 

\begin{subsection}{gameplay}
From the start menu of Teleglitch, the player can choose to play, read a 
tutorial, or choose a level they have previously completed. Once the 
user selects play/continue, they are dropped into the world. Players can move 
the character around the screen using standard keyboard/mouse controls. 

The way levels are designed and generated in Teleglitch is interesting. Not 
only are levels randomly generated each time the user plays, but there is a 
neat mechanic where most of the level remains hidden until your character 
actually sees that section by navigating over to it. This causes great 
moments of surprise when you travel to a new part of the level only to be 
met by a horde of enemies. 

The core of the gameplay comes in the form of killing enemies using various 
weapons. The character begins the game with a sword that he can use to melee 
enemies with. Other weapons like guns and bombs are obtained via loot drops 
from downed enemies and item boxes. One of the main mechanics of the game comes 
in the form of being able to combine certain loot items into more useful weapons. 
For instance, you may find some nails in an item box at one point, and then a bit 
later find a baseball bat after you kill an enemy. The game allows you to simply 
combine these two items into a nail-bat which gives it a damage boost over a 
standard baseball bat. There are multitudes of these potential item combinations. 

The game also incorporates elements of the survival horror genre in the form of 
ammo being somewhat of a rarity, encouraging the player to think very carefully 
about when they choose to use their guns insted of their melee weapons.
\end{subsection}

\begin{subsection}{The Good}
The shooting mechanics in Teleglitch are exceptional. When a gun is fired at an 
enemy, a noticable "blur" effect happens to the character and the bullet itself. 
This effect adds a palpable "weight" to the shots, making them visceral, impactful, 
and highly satisfying. 

The loot combining element is an excellent mechanic. It causes players to think 
critically about the potential ways they can combine the multitude of items in their 
inventory. A great sense of satisfaction comes when you think of a way to combine 
two things and the game allows you to do so exactly as you envisioned. 

The level designs are very cool, having a very industrial, apocalyptic aesthetic. 
The levels are somewhat minimal in design, which lends well to the game overall, as 
there is rarely any unnecessary clutter obstructing you from attacking your foes or 
getting around the environment. 

While it has been a complaint of many industry reviewers, I think the low-fi look of 
the game overall is quite charming. A game like this doesn't need to have super high-res 
graphics. I felt it would add almost nothing to the game at all. 

\end{subsection}

\begin{subsection}{The Bad}
The application of "ammo rarity" in this game only ended up frustrating me, rather than 
giving me a feeling of tension. With Teleglitch, where the shooting is easily the highlight of 
the entire game, implementing a mechanic that lessens the amount of times you will use that 
awesome gunplay seems silly to me. 

The melee combat in Teleglitch is nowhere near as satisfying as the gunplay. There is a noticable 
lack of "feedback" when the melee weapons connect with enemies, sometimes leaving you unsure 
if you actually hit the enemy at all, since most take more than one hit to kill.  

In Teleglitch, when you want to see the whole map of the level you are currently on, you need 
to press a button that actually zooms out from the current room you're in and shows you the whole 
map. I felt this feature broke the immersion of the game. I think an onscreen mini-map that 
could be toggled on and off would have been a much better choice both for Teleglitch and for 
our game.

\end{subsection}

\end{section}

\begin{section}{Binding of Isaac - Lance Wells}

For my section regarding the application domain study, I have decided to review general concepts 
regarding our game's genre in addition to drawing from specific examples introduced in popular 
"Roguelights".

To begin, our general game concept runs closely in the vein of a traditional "Roguelike"; that is, 
our game borrows several features from the genre introduced by the game Rogue. Included in our 
description specifically are:

\begin{itemize}
\item Tile-Based Aesthetic and Generation
\item Dungeon Crawling Gameplay
\item Role-Playing Game thematics
\end{itemize}
To facilitate the description of these items, the following is a list of interpretations of these 
features from the Roguelight "Binding of Isaac".
\begin{itemize}
\item \emph{Tile-Based Aesthetic and Generation}\newline
Binding of Isaac (or more recently, the Rebirth edition), features a tile-based dungeon layout. 
The game limits the range of tile-values to only a select few (Passable Terrain, Impassible Terrain, 
Destructible Terrain, Spikes), while altering tilesets throughout each themed floor set. This strategy 
effectively simplifies the gameplay while still allowing for a large variety of room layouts.
\item	\emph{Dungeon Crawling Gameplay}\newline
Rather than adopting a standard free-movement-by-floor interpretation of a dungeon crawler. Binding 
of Isaac chooses to instead only display one "room" at a time to the player. This effectively lowers 
the complexity in displaying an entire "floor" simultaneously while still maintaining a maze-like 
floor-by-floor dungeon crawler.
\item	\emph{Role-Playing Game thematics}\newline
Binding of Isaac chooses to abandon conventional leveling systems while instead functioning on a 
skill-by-skill basis. Each character is influenced by a series of simple numeric-based "skills" that 
change the gameplay slightly (e.g. movement speed, projectile speed, luck, etc.). Simultaneously, a 
player may alter their experience by picking up various stat-altering or otherwise gameplay-altering 
items ranging from mild to extreme in effect. What more, Binding of Isaac maintains a small portion of 
"risky" decisions often incorporated into roguelikes similar to random-effect pills, varying-effect 
cards, and health-depleting "Deals with the Devil".
\end{itemize}
\end{section}

\begin{section}{Crawl - Morgan Holbart}

I chose to conduct my domain study on the recently released beta version of the game "Crawl." It has some roguelike 
elements, but more closely resembles that of a dungeon crawler. The game is a multiplayer dungeon crawler where 4 
players compete to reach level 10 and defeat the boss before the other players. 

\begin{subsection}{Gameplay}
The game has, as expected for its genre, very simple controls and graphics, the value of the game comes from 
its unique competitive nature and ever changing strategy. The twist to this dungeon crawler is that one player
is a human, and the other 3 players play a the monsters in the dungeon. To start each player chooses a god to 
worship (which gives a passive bonus), and the game begins. One player is randomly selected and sent off to 
explore the dungeon.

Once the player moves from room to room, the other players are given the ability to control monsters in the room
and begin attacking the human player. They are rewarded blood points for damage dealt to the player, and the player
is granted XP for killing monsters. The goal of the human is to try to get as much xp as possible to reach level 10
and try to beat the boss, with careful consideration of the fact that the higher level he gets, when you move on to 
the next dungeon floor, every other player will get points to level up their monsters for every level the human gets.
Once the human is killed, whoever dealt the killing blow will take place as human starting at level 1 (or the previous
level he achieved) and can begin moving on.

All the blood dealt as a monster can be turned to money as a human, and used to buy gear if you find the store that is
on each level. The game balances itself out by making your monster stronger for every level another player gains,
and by only being able to buy gear by earning money while playing as a monster.
\end{subsection}

\begin{subsection}{Good Ideas}
The game is fast paced, there is limited setup and vast replayability. It is focused more on the competitive nature
and goal of winning than it is the roguelike or rpg elements. If every player knows what they are doing the games
can end anywhere between 15-30 minutes with plenty of push-pull for first place.

The game balances itself, if you are a human for a long duration the monsters get buffed, and if you are the monster
for a long duration you will be buffed until you outscale the other monsters and can more easily become the human.

The game is procedurally generated and unpredictable, you never know what the room you are about to enter will have,
it can have an assortment of traps (controllable by monsters also), monsters, and treasure.

The game has a fairly strong AI, that in its hardest difficulty is fairly difficult to beat, and plays the game 
pretty similarly to a human, give or take a couple bumps in the behavior.
\end{subsection}

\begin{subsection}{Bad Ideas}
The game has some balance issues most likely because of the beta state of the game, some monsters are fairly 
overpowered compared to others, and some of the gear can be overpowered. Though careful selection of monster
ugprades can help to counter certain weapons, you can only upgrade if you have points and at the end of each floor,
so the power spike can carry the human until he moves on to the next floor.

The boss can be unbalanced based on the weapons and abilities available to the humans (which are randomly generated).
A ranged weapon trivializes the boss, and a melee weapon is basically impossible to win with against any competent
players because of how the boss is designed.

Level 1 monsters are too weak, and whoever gets human first basically gets free reign over the first dungeon floor 
until people can upgrade the monsters.
\end{subsection}

\begin{subsection}{What Will Translate Well}
The game is a floor based dungeon crawler that is both cooperative and competitive. Players are working together 
as monsters, but against the human, but because of the turn based nature of having control of the human, it
can be difficult to just form alliances retaining the competitive balance. 

The game did procedural generation very well, though sometimes it can be unbalanced if the current human lands 
health potions multiple rooms in a row and another does not.

The game ends fairly quickly, which may or may not be good depending on how in depth we want the RPG elements to be,
if we plan on a fairly low depth RPG system, the game has replayability and speedy gameplay down.

Unlockables, based on the length of the game, you gain XP which unlocks new items, gods, and monsters for the game,
this adds to the replayability, and allows you to add strategic depth over time, keeping the game simple to start
and increasing the strategy as you play more and more. Also gives you a reason to play again (try a new item)
\end{subsection}
\end{section}

\begin{section}{Wanderlust: Rebirth - Tessa Saul}
I chose to look at the tile-based RPG Wanderlust: Rebirth. I chose this game because of the games I already 
had, this one best fit the discription of what we are trying to do. This game has a single-player mode and 
a co-op mode with up to four people. 

I played through the tutorial and first couple levels in single-player mode. 

\begin{subsection}{The Good}
This game has a simple and quick class selection system. There are four classes 
and two genders. Each combination has a detailed portrait. Besides that, the 
only thing you have to pick to make a new character is a name. 
This makes starting a new co-op game very fast.  
Leveling up was also very easy and fast. The game gives you points to distribute 
among your skills. Depending on how you place them, you could 
probably have very different builds. 
The game has a `lobby' where you can reassign all of the points that you have
earned into different builds, and then try them out. 
Another thing I liked about the game is that when I died multiple times, 
my character was stunned for a few seconds, and then I got to continue playing. 
I didn't even realize that I died until the end of the level when it deducted points 
for the death. This means that even if you are very bad at the game, it will not interrupt
the fun you are having to point out how bad you are. 
The game fills in your abilities by giving you AI companions as well. This is nice because
you do not have to play a class to have its abilities, like healing. 

\end{subsection}
\begin{subsection}{The Bad}
The inventory and crafting systems in this game feel very clunky. The tutorial had 
me create an item from a recipe, but I had to select all the items for the recipe by hand in my inventory. 
This was frustrating because the items only had little icon type images to represent them,
so I had to look at each item to see if it was one I needed for the recipe. If we have an inventory system,
I think we should look at other games for ideas. 

The controls on the game feel `spammy'. By this, I mean that you can pretty much press random buttons rapidly
and get about the same result as if you planned out an attack. This is emphasized by the fact that if you play
single-player, the game gives you three AI companions that are the other classes. They swarm around you 
when you hold still, and attack everything at the same time as you, so it is hard to see what you are doing. 

\end{subsection}
\end{section}

\begin{section}{Quest of Dungeons - Zachary Yama}
For my application domain study I chose to analyze the dungeon crawler, Quest of Dungeons. 
I chose this title because it’s a more recent, indie approach to the genre of dungeon crawlers 
and we will be approaching our game from the same sort of perspective. It also happens to be 
fairly popular, so it would be interesting to see what features make this game a successful 
dungeon crawler.
 
The basis of this study will be drawn from a single 15 minute campaign session. Throughout the 
session I will record any important game features and interesting experiences I find. Once a 
reasonable list has been compiled I will break it into three sections: how the gameplay directs 
the player, what elements could be included in our game, and elements that probably shouldn’t 
be included in our game.
 
\begin{subsection}{Gameplay and Direction}
On game launch, the player is presented with a generic and straightforward navigation menu. 
Once a campaign is started, the player is asked to pick a class. There are four choices: wizard, 
warrior, assassin, and shaman. After a class is selected, a short humorous cut scene is presented 
where the other three classes that were not selected suggest you take on the world’s most evil 
villain and plunge into his dungeon alone. Reluctantly the class that was chosen by the player 
exits the scene and the actual game begins. In my case, it was the Wizard.
 
After being dropped into the game, I noticed a generic set of GUI elements: health, mana, a 
minimap, a place to set skills for quick use, and a set of buttons to open the inventory, stats, 
and quest menus. After playing with the keyboard for a bit all the movement controls become clear. 
The game uses WASD movement and a point and click system to perform actions.
 
Now that I had an idea of how the mechanics worked I decided to go see if I could kill something. 
In fact I could; with giant fireballs. I also noticed there were many breakable objects lying 
around. Many yielded gold and even items and spells I could equip. After running around for a 
while I found a large stone that began a quest to kill so and so with some neatly included 
directions. After spending about five minutes heading off on an interesting adventure I decided 
to be done and quit the game.
 
\end{subsection}
 
\begin{subsection}{Useful Features}
Many of the features I experienced in Quest of Dungeons were really fun and interesting. Though, 
with such a short session my first impressions may not do all elements justice. Regardless, there 
are some elements that would blend well in our game, and others not so much.
 
The most interesting thing I came across was the spell system. There aren’t any skills, you have 
to find them from drops. What’s particularly interesting is that you can use spells that seem more 
like something a warrior would use. The catch is that the spell does damage scaled off of a specific 
stat. For example, the fireball skill I had scaled off of intellect by a rating of B, where the rest 
of the stats it scaled off of were F.
 
Another game element I found inspirational was the fast paced combat system. Monsters do only move 
when you do, but the way the game behaved was very polished and quick. This is somewhat attributed 
to the WASD movement and the snappy response of having an active spell that is used only by pointing 
and clicking. I didn’t need to select anything or think about what to do; I just did it.

Lastly, the questing system was somewhat interesting. Being able to go out adventuring and randomly 
find a stone that gives you newfound direction was rather refreshing. It would be nice to include some
sort of questing system in our game so that the player isn’t suck with just running around the dungeon 
waiting for the last level to come.  
 
\end{subsection}
 
\begin{subsection}{Harmful Features}
This game was extremely fun to play, but my experiences were with a single player campaign. There are 
some components that clearly wouldn’t work for our multiplayer version, which is what we intend to do.
 
One of these features is locking a player in a room. For a multiplayer system this would be more harmful 
than beneficial. If a party accidently overextends into a higher level room, they would likely die and 
lose valuable experience and have to restart at no real fault of their own. There would also be 
inconvenient communication requirements to make sure that the whole party is ready to move on, 
and which room they decide as a whole to move to next. These two elements will needlessly slow down 
the players.
 
Another feature that shouldn’t be included is permanent death. If a player dies, they are required to 
start completely over. This would prevent players from wanting to play with each other simply because 
they don’t know how prepared the other player really is. This again slows party progress and requires 
additional unnecessary communication.
 
\end{subsection}
\end{section}

\begin{section}{Player Tasks} % Zack Y; I think this was supposed to be here, I'm not sure.

moving through areas % Tessa Saul 

Actor: Player

Preconditions: The player has the area being moved to unlocked or available to him.

Summary: The player accepts a prompt moving him to a new dungeon, floor, or back to the lobby.

Steps:
1. The player uses an object (door/computer).
2. Player hits yes to the prompt on whether or not to use it.
3. The player is moved from the current location to the new location.
4. The new level is constructed (if a dungeon), or loaded (if the lobby) and the player is placed in it.
------------------
crafting items % Morgan Holbart

Actor: Player

Precondition: The player has the necessary ingredients to make a crafted item

Summary: The PLayer takes ingredients for a crafted item and expends them to recieve a different item

Steps:
1. Player opens crafting menu
2. Player selects crafting recipe
3. The items for the recipe are destroyed
4. The crafted item is placed in the player's inventory
------------------
selling items % Tessa Saul 

Actor: Shopkeeper and player

Preconditions: The player has an item that the shopkeeper is willing to buy. The shopkeeper has sufficient
currency for the transaction

Summary: Player exchanges an item for currency with a shopkeeper

Steps: 
1. Player approaches shopkeeper
2. Player engages shop
3. Player selects item from their inventory to sell
4. Item is given to shopkeeper, currency is taken from shopkeeper and given to player
------------------
buying items % Tessa Saul 

Actor: Shopkeeper and player

Preconditions: The shopkeeper has an item that the player is willing to buy. The player has sufficient
currency for the transaction

Summary: Player exchanges an currency for item with a shopkeeper

Steps: 
1. Player approaches shopkeeper
2. Player engages shop
3. Player selects item from shopkeeper's inventory to buy
4. Item is given to player, currency is taken from player and given to shopkeeper
------------------
picking up items % Tessa Saul 

Actor: Player

Precondition: The player has an empty slot big enough for the item in their inventory

Summary: A player's action transfers an item from the environment into their inventory

Steps:
1. Player approaches item
2. Player engages item
3. Item is removed from environment
4. Item is placed in the player's  inventory
------------------
dialog % Morgan Holbart

Actors: Player and NPC

Preconditions:
1. The player is currently able to interact with the target
2. The target wishes to interact with the player
3. The NPC has something to say

Summary: A player tries to initiate conversation with an NPC

Steps:
1. The player interacts with the NPC
2. The NPC responds to the player
3. The player can respond based on user input
4. When the NPC runs out of things to say, the dialog ends
------------------
attacking % Morgan Holbart

Actor: Player

Preconditions:
1. Player has a valid weapon to use equipped
2. Player has ammo or durability to use equipped weapon
3. Player is able to attack (not in a current dialog/other interaction)

Steps:
1. The player presses the input to attack
2. Ammunition/Durability is subtracted
3. The weapon animation/effects/etc are instantiated
------------------
opening doors
------------------
opening inventory/closing inventory % Morgan Holbart

Actor: Player

Preconditions: 
1. Player can currently perform the action of opening/closing inventory

Steps:
1. The player presses the input to open/close the inventory
2. The inventory is loaded and displayed or closed 
------------------
starting game % Morgan Holbart

Actor: User

Preconditions: 
1. Player is in the menu
2. Player is selecting the start game button

Steps:
1. The player presses the input to start the game
2. The game loads the user save data
3. The game loads the lobby level with regard to the user save data
------------------
Reload Weapon? % Morgan Holbart

Actor: Player

Preconditions: 
1. Player has a weapon equipped that can reload
2. Player has ammo to reload with

Steps:
1. Press the reload button
2. Subtract ammo from the stored ammo
3. Add the subtracted ammo to the current ammo
-----------------
Start Dungeon % Morgan Holbart

Actor : Dungeon Leader(Player)

Preconditions:
1. Player is the leader of a dungeon group
2. Player is currently connected with all party members
3. Party members have all selected the ready button
4. Party members all have adequate reputation to do the dungeon

Steps:
1. Player interacts with the computer he wishes to enter
2. All other players are in party already, or join the party through the computer
3. The party leader presses the start dungeon button.
----------------

\end{section}




\end{document}
