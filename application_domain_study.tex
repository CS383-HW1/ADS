\documentclass[12pt]{report}
\usepackage{geometry}
\usepackage{titling}

\geometry{margin=1in}

\newcommand{\subtitle}[1]{%
	\posttitle{%
		\par\end{center}
		\begin{center}\large#1\end{center}
		\vskip0.5em}%
}

\title{Application Domain Study}
\subtitle{CS 383 - Homework 1 - Group 3}
\author{Mason Fabel, NAMES} % TODO: Everyone add your name to this list
\date{\today}

\begin{document}

\maketitle

\chapter{Related Applications}

% TODO: Each person needs a section here.

\begin{section}{Nethack - Mason Fabel}
For the application domain study, I chose to study the classic roguelike
game Nethack. I chose this title for a number of reasons. First of all,
the roguelike genre is, if not the direct predecesor, the inspiration for
the majority of tile-based dungeon crawling games, among others. This being
so, I chose Nethack as it a mature and heavily played roguelike, being one
of a handful of big-name roguelike games.

For my study, I am going to play a session of Nethack and record a
condensed version of my adventures. I will then discuss the elements that
I think can be used successfully in our project, followed by the elements
that I think can not.

\begin{subsection}{Gameplay}
Nethack opens with a pretty straightforward selection of menus asking the
player for their class, race, and gender. I selected a human male Tourist,
as I am merely making a small tour of this game for this study. A short,
three paragraph intro is dropped on me, telling me to descend the
dungeon and fetch some amulet.

And that's where the handholding stops. I'm dropped into the game with the
oh-so-helpful message "Be careful!", and nothing else. Some quick fiddling
will reveal that the various keypresses correspond to different actions,
and that pressing '?' will bring up some very cryptic help. Fortunately,
this is far from my first time in Nethack, so I'm over the gigantic
learning curve needed to even make sense of the screen and perform basic
actions.

Thus, off I go, looking for the amulet. The game itself was pretty simple
to watch. Explore the map, look for the stairs to go deeper, run away from
monsters until my pet kills them and I loot the corpses. In this manner I
cleared a few levels and then called it a day.
\end{subsection}

\begin{subsection}{The Good}
\end{subsection}

\begin{subsection}{The Bad}
\end{subsection}

\end{section}

\end{document}
