\documentclass[12pt]{report}
\usepackage{geometry}
\usepackage{titling}

\geometry{margin=1in}

\newcommand{\subtitle}[1]{%
	\posttitle{%
		\par\end{center}
		\begin{center}\large#1\end{center}
		\vskip0.5em}%
}

\title{Application Domain Study}
\subtitle{CS 383 - Homework 1 - Group 3}
\author{Mason Fabel, NAMES} % TODO: Everyone add your name to this list
\date{\today}

\begin{document}

\maketitle

\chapter{Related Applications}

% TODO: Each person needs a section here.

\begin{section}{Nethack - Mason Fabel}
For the application domain study, I chose to study the classic roguelike
game Nethack. I chose this title for a number of reasons. First of all,
the roguelike genre is, if not the direct predecesor, the inspiration for
the majority of tile-based dungeon crawling games, among others. This being
so, I chose Nethack as it a mature and heavily played roguelike, being one
of a handful of big-name roguelike games.

For my study, I am going to play a session of Nethack and record a
condensed version of my adventures. I will then discuss the elements that
I think can be used successfully in our project, followed by the elements
that I think can not.

\begin{subsection}{Gameplay}
\end{subsection}

\begin{subsection}{The Good}
\end{subsection}

\begin{subsection}{The Bad}
\end{subsection}

\end{section}

\end{document}
